\documentclass{article}
\usepackage{mathtools}
\begin{document}
\[
THP_{a,L}=\frac{88}{33000}\left[\frac{\sigma A_D V^3}{391}+\frac{391\left(\frac{W}{b_e}\right)^2}{\pi \sigma V}\right]
\]
\[
THP_{a,L}=\frac{88}{33000}\left[\frac{\sigma A_D V^2}{391}+\frac{391\left(\frac{W}{b_e}\right)^2}{\pi \sigma (V)^2}\right]\cdot V
\]
\[
THP_{a,L}=\frac{88}{33000}\left[\frac{\sigma A_D V^2}{391}+\frac{391W^2}{\pi \sigma V^2{b_e}^2}\right]\cdot V
\]
Which is similar to Formulas 23, but where to from here to obtain a fundamental understanding of its origin?
\[
THP_{a,L}=\frac{88}{33000}\left[DS+\frac{391W^2}{\pi \sigma V^2{b_e}^2}\right]\cdot V
\]
\text{From Relation 5:}
\begin{align*}
THP_a&=\frac{A_D {V_{max}}^3}{146625}\\
&=\frac{A_D {V_{max}}^2}{146625} V_{max}
\end{align*}
Given that air pressure ratio at sea level = 1 and,
\[
\frac{1}{146625}v = \frac{88}{33000}\frac{1}{391}
\]
\[
THP_a=\frac{88}{33000}\left[\frac{\sigma A_D {V_{max}}^2}{391}\right]V
\]
\[
A_D = C_{D,0}S
\]
\[
THP_a=\frac{88}{33000}\left[\frac{\sigma\ C_{D,0}\ S\ {V_{max}}^2}{391}\right]V
\]
\end{document}
